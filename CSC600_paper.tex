\documentclass[10pt,conference]{IEEEtran}
\IEEEoverridecommandlockouts
% The preceding line is only needed to identify funding in the first footnote. If that is unneeded, please comment it out.
\usepackage{balance}
\usepackage{xspace}
\usepackage{graphicx}
\usepackage[mathscr]{euscript}
\usepackage{listings}
\usepackage{enumitem}
\usepackage{multirow}
\usepackage{amsmath,amssymb}
\usepackage[usenames, dvipsnames]{color}
\usepackage{wrapfig}
\usepackage{cite}
\usepackage{float}
\usepackage[flushleft]{threeparttable}
\usepackage{pifont}
\usepackage{scrextend}
\usepackage{soul}
\usepackage{color} 
\usepackage{fancyvrb}
\usepackage{hyperref}
\usepackage{lettrine}


\def\BibTeX{{\rm B\kern-.05em{\sc i\kern-.025em b}\kern-.08em
    T\kern-.1667em\lower.7ex\hbox{E}\kern-.125emX}}
\begin{document}

\title{Document Refactorings}

\author{
\IEEEauthorblockN{Soumya Mudiyappa}
\IEEEauthorblockA{\textit{West Chester University}\\fs926226@wcupa.edu}
\and
\IEEEauthorblockN{Swetchha Shukla}
\IEEEauthorblockA{\textit{West Chester University}\\ss928947@wcupa.edu}
\and
\IEEEauthorblockN{Yung-Chen Cheng}
\IEEEauthorblockA{\textit{West Chester University}\\yc917559@wcupa.edu}
}

\maketitle

\section{\textbf{Introduction}}
\lettrine{R}{efactoring} is a controlled technique for improving the design of an existing code base. Its essence is applying a series of small behavior-preserving transformations, each of which "too small to be worth doing". However the cumulative effect of each of these transformations is quite significant. By doing them in small steps we reduce the risk of introducing errors. We also avoid having the system broken while carrying out the restructuring - which allows us to gradually refactor a system over an extended period of time.~\cite{Fowler}

In simple words, refactoring is a \emph{behavior preserving} code transformation technique, for example- rename, move, extract etc. Refactoring rely on two important factors - \emph{precondition checks and code changes}.

The idea behind refactoring were introduced over decades ago, but scripting of refactorings is poorly suppported by current IDEs tools. Today's IDEs lack fine-grained primitive refactorings including restrictive/imprecise preconditions and inconsistent code transformations. 
There are further more challenges for executing refactoring as it is error-prone, time-consuming and laborious work. Scripting refactoring with current IDEs are too slow because precondition checks and AST operation takes largest time which makes scripting refactorings impractical.
Therefore, we need tools that automate refactorings scripts with precise preconditions, high speed and reliability and which we may reuse frequently.

\section{\textbf{Rename Refactorings}}
Rename refactoring is a feature that provides an easy way to change the name of identifiers for code symbols without changing its behavior and functionality. Rename can be used to change the names in comments and in strings and to change the declarations and calls of an identifier.
There are five types of Rename Refactoring: 
\begin{enumerate}
	\item Rename Class Declarations 
	\item Rename Method Declarations  
	\item Rename Field Declarations  
	\item Rename Local Variables  
	\item Rename Package Declarations
\end{enumerate}

\section{\textbf{Preconditions of Rename Class Refactoring}}
Rename Class Declarations is a refactoring feature that changes the name of the class and all references to that class to the new name without changing its behavior and functionality. There are certain preconditions required for Rename Class Refactoring. 
\begin{enumerate}
	\item The target class cannot be duplicate with an existing class within same package after rename.
	\item The target class cannot be duplicate with any imported class from another package.
	\item The target sub-class cannot be duplicate with any imported class into its parent class from another package. 
\end{enumerate}

\subsection{The target class cannot be duplicate with an existing class within same package after rename.}

When we try to rename a class with an existing class name, the Eclipse produces syntax error:
``Please choose another name".~\cite{EclipseWebPage} The classes will be conflicted if we rename the target class using the name of an existing class in the same package. So we can not have duplicate class names in the same package. 

For example, we want to refactor the class name \textsl{A} to \textsl{B} as fig. \ref{fig:afterrr}, then the java compiler shows up the error that B.java already exists as fig. \ref{fig:renameclassname}.

\begin{figure}[th]
\centering
\begin{minipage}[t]{0.45\linewidth}
\begin{lstlisting}[language=java, basicstyle=\scriptsize\ttfamily,frame=single]
package p;

class A{
}
	
class B{
}

class C{
}
 
\end{lstlisting}
\tiny{(a) Before}
\end{minipage}
\hfill
\begin{minipage}[t]{0.45\linewidth}
\begin{lstlisting}[language=java, basicstyle=\scriptsize\ttfamily,frame=single]
package p;

class B{
}	

class B{
}

class C{
}

\end{lstlisting}
\tiny{(b) After}
\end{minipage}
\caption{Example of Rename Class Refactoring from A to B}
\label{fig:afterrr}
\end{figure}

\begin{figure}[H]
\centerline{\includegraphics[width=85mm,scale=0.5]{SCN.jpg}}
\caption{The error of using same class name for refactoring}
\label{fig:renameclassname}
\end{figure}

Furthermore, this precondition is applicable to nested classes. The examples below show that we can not use the same name either as inner or as outer class for nested classes:

\begin{figure}[th]
\centering
\begin{minipage}[t]{0.75\linewidth}
\begin{lstlisting}[language=java, basicstyle=\scriptsize\ttfamily,frame=single]
package p;

public class A{	

  class M{
  }

  class N{
  }
} 
\end{lstlisting}
\end{minipage}
\caption{Original Nested Class}
\label{fig:original}
\end{figure}
\begin{itemize}
\item Example 1: The rename refactoring of the inner class can not be the same name as other inner classes' name. When we try to rename the inner class M to N as fig. \ref{fig:nestedclass1}, the java compiler shows up the error as fig. \ref{fig:NC1}.
\end{itemize}

\begin{figure}[th]
\centering
\begin{minipage}[t]{0.75\linewidth}
\begin{lstlisting}[language=java, basicstyle=\scriptsize\ttfamily,frame=single]
package p;

public class A{	
    
  class N{
  }
    
  class N{
  }
} 
\end{lstlisting}
\end{minipage}
\caption{Example 1 of Nested Class Rename Refactoring}
\label{fig:nestedclass1}
\end{figure}

\begin{figure}[H]
\centerline{\includegraphics[width=85mm,scale=0.5]{NC1.jpg}}
\caption{The error of duplicate inner class name for refactoring}
\label{fig:NC1}
\end{figure}

\begin{itemize}
\item Example 2: The rename refactoring of the outer class can not be the same name as the inner classes' name, vice versa. When we try to either rename outer class to inner class name or rename inner class to outer class name, the java compiler shows up the error as fig. \ref{fig:NC2} and fig. \ref{fig:NC3}.
\end{itemize}

\begin{figure}[th]
\centering
\begin{minipage}[t]{0.45\linewidth}
\begin{lstlisting}[language=java, basicstyle=\scriptsize\ttfamily,frame=single]
package p;

public class M{	
  
  class M{
  }
	
  class N{
  }
} 
\end{lstlisting}
\tiny{(a) Rename Outer Class}
\end{minipage}
\hfill
\begin{minipage}[t]{0.45\linewidth}
\begin{lstlisting}[language=java, basicstyle=\scriptsize\ttfamily,frame=single]
package p;

public class A{	
    
  class A{
  }
    
  class N{
  }
} 
\end{lstlisting}
\tiny{(b) Rename Inner Class}
\end{minipage}
\caption{Example 2 of Nested Class Rename Refactoring}
\label{fig:nestedclass2}
\end{figure}

\begin{figure}[H]
\centerline{\includegraphics[width=85mm,scale=0.5]{NC2.jpg}}
\caption{The error of renaming outer class as inner class name}
\label{fig:NC2}
\end{figure}

\begin{figure}[H]
\centerline{\includegraphics[width=85mm,scale=0.5]{NC3.jpg}}
\caption{The error of renaming inner class as outer class name}
\label{fig:NC3}
\end{figure}


Also, this precondition is applicable even if one or the other file is empty. So checking whether a class with the same name already exists in a package should be the first job we have to do for RcR. 
   
\label{sec:precon1}
	
\subsection{The target class cannot be duplicate with any imported class from different package after rename.}

If a class is imported from another package, we have to pre-check that the target class meant to be rename refactored does not match with the imported class. This is because the compiler will not be able to distinguish between the imported class and the existing class. 

\begin{figure}[th]
\centering
\begin{minipage}[t]{0.6\linewidth}
\begin{lstlisting}[language=java, basicstyle=\scriptsize\ttfamily,frame=single]
package beforevisitor;
class A{
    
}	
class Test{

}

package aftervisitor;
import beforevisitor.Test;
class A{

}
class B{

} 

\end{lstlisting}
\tiny{(a) Before Rename Refactoring Class B}
\end{minipage}
\hfill

\begin{minipage}[t]{0.6\linewidth}
\begin{lstlisting}[language=java, basicstyle=\scriptsize\ttfamily,frame=single]
package beforevisitor;
class A{
    
}	
class Test{

}

package aftervisitor;
import beforevisitor.Test;
class A{

}
class Test{

} 
\end{lstlisting}
\tiny{(b) After Rename Refactoring Class B to Test}
\end{minipage}
\caption{Precondition when importing a class}
\label{figure:fig2}
\end{figure}

In Fig. \ref{figure:fig2} (a), we see that class B is not a duplicate for class A and class B can be rename refactored to any other name instead of A as mentioned in section \ref{sec:precon1}. However, in Fig. \ref{figure:fig2} (b), when we try to rename refactor the class B to class Test, we get compile error \textit{"a compilation unit must not import and declare a type with the same name"}~\cite{EclipseWebPage}. 
This is because a class named Test is being imported from another package and on renaming the target class B to Class Test will have compilation error due to duplicate names. 

\begin{figure}[th]
\centering
\begin{minipage}[t]{0.75\linewidth}
\begin{lstlisting}[language=java, basicstyle=\scriptsize\ttfamily,frame=single]
package aftervisitor;
public class Visitor {
  public void print() {
      System.out.println("Hello");
  }
}

package beforevisitor;
import aftervisitor.Visitor;
class A {
    public static void main(String[] args) {
        Visitor v = new Visitor();
        v.print();
    }
}

\end{lstlisting}
\tiny{(a) Before Rename Refactoring Class A}
\end{minipage}
\hfill

\begin{minipage}[t]{0.75\linewidth}
\begin{lstlisting}[language=java, basicstyle=\scriptsize\ttfamily,frame=single]
package aftervisitor;
public class Visitor {
  public void print() {
      System.out.println("Hello");
  }
}

package beforevisitor;
import aftervisitor.Visitor;
class Visitor {
    public static void main(String[] args) {
        Visitor v = new Visitor();
        v.print();
    }
}
\end{lstlisting}
\tiny{(b) After Rename Refactoring Class A to Visitor}
\end{minipage}
\caption{Class already defined in Compilation Unit}
\label{figure:error}
\end{figure}


\begin{figure}[H]
\centerline{\includegraphics[width=85mm,scale=0.5]{CUE.jpg}}
\caption{The Compilation Unit Error}
\label{figure:pic}
\end{figure}

Another example where rename refactoring requires a pre-check if the target name is not already defined in the compilation unit is shown below . In Fig. \ref{figure:error} (a), on executing the output generated is \textit{Hello}. However, after rename refactoring class A to Visitor as shown in Fig. \ref{figure:error} (b), we get compile error as shown in Fig. \ref{figure:pic}.
\\Therefore, it is essential to pre-check that the target class should not have duplicate name with any of imported class after rename refactoring class. 




\label{sec:precon2}

\subsection{If a parent class imports a class from different package, the target child class within same java file cannot be duplicate with that imported class after rename.}

If a parent class imports a class from different package and if we try renaming a child class with the same name as the class imported into its parent class within the same java file, compiler produces an error as \textit{``a compilation unit must not import and declare a type with the same name''}~\cite{EclipseWebPage}.This precondition can be explained by the following example.

\begin{figure}[th]
\centering
\begin{minipage}[t]{0.45\linewidth}
\begin{lstlisting}[language=java, basicstyle=\scriptsize\ttfamily,frame=single]	
A.java

package q;
import p.C;

public class A{	
}

class B extends A{	
}

class D extends B{
}
\end{lstlisting}
\centering{(a) Before}
\end{minipage}
\hfill
\begin{minipage}[t]{0.45\linewidth}
\begin{lstlisting}[language=java, basicstyle=\scriptsize\ttfamily,frame=single]
A.java

package q;
import p.C;

public class A{	
}

class C extends A{	
}

class D extends B{
}	
\end{lstlisting}
\centering{(b) After}
\end{minipage}
\caption{\textbf{RcR from B to C}}
\label{figure:figpc3_1}
\end{figure}

From the above Fig. \ref{figure:figpc3_1}, we see that if a parent class A imports a class C from package `p' and if we try to rename the child class from B to C or D to C, Java compiler produces an error as shown in the Fig. \ref{figure:figpc3_2}. As mentioned in  section \ref{sec:precon2}, the same precondition also holds good for renaming a child class. This precondition is applicable for all ancestor class and we have to trace back and check if any of the parent class is importing a class with same name before renaming the child class within the same java file.
\begin{figure}[htbp]
\centerline{\includegraphics[width=85mm,scale=0.5]{precond3.png}}
\caption{\textbf{Compile error for RcR child class to imported class}}
\label{figure:figpc3_2}
\end{figure}

If a parent class imports a class from different package and if the child class is defined in a separate java file, then in that case we can refactor and rename the child class to the imported class name.


\section{\textbf{Code Change Rules}}


\bibliographystyle{IEEEtran}
\bibliography{ref}

\end{document}
