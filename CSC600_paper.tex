\documentclass[10pt,conference]{IEEEtran}
\IEEEoverridecommandlockouts
% The preceding line is only needed to identify funding in the first footnote. If that is unneeded, please comment it out.
\usepackage{balance}
\usepackage{xspace}
\usepackage{graphicx}
\usepackage[mathscr]{euscript}
\usepackage{listings}
\usepackage{enumitem}
\usepackage{multirow}
\usepackage{amsmath,amssymb}
\usepackage[usenames, dvipsnames]{color}
\usepackage{wrapfig}
\usepackage{cite}
\usepackage{float}
\usepackage[flushleft]{threeparttable}
\usepackage{pifont}
\usepackage{scrextend}
\usepackage{soul}
\usepackage{color} 
\usepackage{fancyvrb}
\usepackage{hyperref}
\usepackage{lettrine}
\usepackage[T1]{fontenc}


\def\BibTeX{{\rm B\kern-.05em{\sc i\kern-.025em b}\kern-.08em
    T\kern-.1667em\lower.7ex\hbox{E}\kern-.125emX}}
\begin{document}

\title{Preconditions on Rename Refactorings}

\author{
\IEEEauthorblockN{Soumya Mudiyappa}
\IEEEauthorblockA{\textit{West Chester University}\\fs926226@wcupa.edu}
\and
\IEEEauthorblockN{Swetchha Shukla}
\IEEEauthorblockA{\textit{West Chester University}\\ss928947@wcupa.edu}
\and
\IEEEauthorblockN{Yung-Chen Cheng}
\IEEEauthorblockA{\textit{West Chester University}\\yc917559@wcupa.edu}
}

\maketitle

\section{\textbf{Introduction}}
\lettrine{R}{efactoring} is a controlled technique for improving the design of an existing code base. Its essence is applying a series of small behavior-preserving transformations, each of which ``too small to be worth doing''. However the cumulative effect of each of these transformations is quite significant. By doing them in small steps we reduce the risk of introducing errors. We also avoid having the system broken while carrying out the restructuring - which allows us to gradually refactor a system over an extended period of time.~\cite{Fowler}

In simple words, refactoring is a behavior preserving code transformation technique, for example- rename, move, extract etc. Refactoring rely on two important factors - precondition checks and code changes. In this paper, we explain about the rename refactoring and the preconditions in detail for each type of rename refactorings. 

The idea behind refactoring were introduced over decades ago, but scripting of refactorings is poorly suppported by current IDEs tools. Today’s IDEs lack fine-grained primitive refactorings including restrictive/imprecise preconditions and inconsistent code transformations. There are further more challenges for executing refactoring as it is error-prone, timeconsuming and laborious work. Scripting refactoring with current IDEs are too slow because precondition checks and AST operation takes largest time which makes scripting refactorings impractical. Therefore, we need tools that automate refactorings scripts with precise preconditions, high speed and reliability and which we may reuse frequently. 

\section{\textbf{Rename Refactorings}}
Rename Refactoring changes the name of identifiers in a program without changing the program's behavior.
There are five types of Rename Refactoring in Java and each of them have different preconditions. 
\begin{enumerate}
	\item Rename Class Declarations 
	\item Rename Method Declarations  
	\item Rename Field Variables
	\item Rename Local Variables 
	\item Rename Package Declarations
\end{enumerate}

\section{\textbf{Preconditions of Rename Class Refactoring}}
Rename Class Refactoring (RcR) changes the name of the class and all references to that class to the new name without changing its behavior. There are certain preconditions required for RcR. 
\begin{enumerate}
	\item The target class cannot be duplicate with any existing class within same package after rename.
	\item The target class cannot be duplicate with any imported class from different package after rename.
	\item The target class name cannot be duplicate with any existing java file name within same package after rename.
\end{enumerate}

\subsection{The target class cannot be duplicate with an existing class within same package after rename.}

When we try to rename a class with an existing class name, the Eclipse produces syntax error:
``Please choose another name".~\cite{EclipseWebPage} The classes will be conflicted if we rename the target class using the name of an existing class in the same package. So we can not have duplicate class names in the same package. 

For example, we want to refactor the class name \textsl{A} to \textsl{B} as fig. \ref{fig:afterrr}, then the java compiler shows up the error that B.java already exists as fig. \ref{fig:renameclassname}.

\begin{figure}[th]
\centering
\begin{minipage}[t]{0.45\linewidth}
\begin{lstlisting}[language=java, basicstyle=\scriptsize\ttfamily,frame=single]
package p;

class A{
}
	
class B{
}

class C{
}
 
\end{lstlisting}
\tiny{(a) Before}
\end{minipage}
\hfill
\begin{minipage}[t]{0.45\linewidth}
\begin{lstlisting}[language=java, basicstyle=\scriptsize\ttfamily,frame=single]
package p;

class B{
}	

class B{
}

class C{
}

\end{lstlisting}
\tiny{(b) After}
\end{minipage}
\caption{Example of Rename Class Refactoring from A to B}
\label{fig:afterrr}
\end{figure}

\begin{figure}[H]
\centerline{\includegraphics[width=85mm,scale=0.5]{SCN.jpg}}
\caption{The error of using same class name for refactoring}
\label{fig:renameclassname}
\end{figure}

Furthermore, this precondition is applicable to nested classes. The examples below show that we can not use the same name either as inner or as outer class for nested classes:

\begin{figure}[th]
\centering
\begin{minipage}[t]{0.75\linewidth}
\begin{lstlisting}[language=java, basicstyle=\scriptsize\ttfamily,frame=single]
package p;

public class A{	

  class M{
  }

  class N{
  }
} 
\end{lstlisting}
\end{minipage}
\caption{Original Nested Class}
\label{fig:original}
\end{figure}
\begin{itemize}
\item Example 1: The rename refactoring of the inner class can not be the same name as other inner classes' name. When we try to rename the inner class M to N as fig. \ref{fig:nestedclass1}, the java compiler shows up the error as fig. \ref{fig:NC1}.
\end{itemize}

\begin{figure}[th]
\centering
\begin{minipage}[t]{0.75\linewidth}
\begin{lstlisting}[language=java, basicstyle=\scriptsize\ttfamily,frame=single]
package p;

public class A{	
    
  class N{
  }
    
  class N{
  }
} 
\end{lstlisting}
\end{minipage}
\caption{Example 1 of Nested Class Rename Refactoring}
\label{fig:nestedclass1}
\end{figure}

\begin{figure}[H]
\centerline{\includegraphics[width=85mm,scale=0.5]{NC1.jpg}}
\caption{The error of duplicate inner class name for refactoring}
\label{fig:NC1}
\end{figure}

\begin{itemize}
\item Example 2: The rename refactoring of the outer class can not be the same name as the inner classes' name, vice versa. When we try to either rename outer class to inner class name or rename inner class to outer class name, the java compiler shows up the error as fig. \ref{fig:NC2} and fig. \ref{fig:NC3}.
\end{itemize}

\begin{figure}[th]
\centering
\begin{minipage}[t]{0.45\linewidth}
\begin{lstlisting}[language=java, basicstyle=\scriptsize\ttfamily,frame=single]
package p;

public class M{	
  
  class M{
  }
	
  class N{
  }
} 
\end{lstlisting}
\tiny{(a) Rename Outer Class}
\end{minipage}
\hfill
\begin{minipage}[t]{0.45\linewidth}
\begin{lstlisting}[language=java, basicstyle=\scriptsize\ttfamily,frame=single]
package p;

public class A{	
    
  class A{
  }
    
  class N{
  }
} 
\end{lstlisting}
\tiny{(b) Rename Inner Class}
\end{minipage}
\caption{Example 2 of Nested Class Rename Refactoring}
\label{fig:nestedclass2}
\end{figure}

\begin{figure}[H]
\centerline{\includegraphics[width=85mm,scale=0.5]{NC2.jpg}}
\caption{The error of renaming outer class as inner class name}
\label{fig:NC2}
\end{figure}

\begin{figure}[H]
\centerline{\includegraphics[width=85mm,scale=0.5]{NC3.jpg}}
\caption{The error of renaming inner class as outer class name}
\label{fig:NC3}
\end{figure}


Also, this precondition is applicable even if one or the other file is empty. So checking whether a class with the same name already exists in a package should be the first job we have to do for RcR. 
   
\label{sec:precon1}
	
\subsection{The target class cannot be duplicate with any imported class from different package after rename.}

If a class is imported from another package, we have to pre-check that the target class meant to be rename refactored does not match with the imported class. This is because the compiler will not be able to distinguish between the imported class and the existing class. 

\begin{figure}[th]
\centering
\begin{minipage}[t]{0.6\linewidth}
\begin{lstlisting}[language=java, basicstyle=\scriptsize\ttfamily,frame=single]
package beforevisitor;
class A{
    
}	
class Test{

}

package aftervisitor;
import beforevisitor.Test;
class A{

}
class B{

} 

\end{lstlisting}
\tiny{(a) Before Rename Refactoring Class B}
\end{minipage}
\hfill

\begin{minipage}[t]{0.6\linewidth}
\begin{lstlisting}[language=java, basicstyle=\scriptsize\ttfamily,frame=single]
package beforevisitor;
class A{
    
}	
class Test{

}

package aftervisitor;
import beforevisitor.Test;
class A{

}
class Test{

} 
\end{lstlisting}
\tiny{(b) After Rename Refactoring Class B to Test}
\end{minipage}
\caption{Precondition when importing a class}
\label{figure:fig2}
\end{figure}

In Fig. \ref{figure:fig2} (a), we see that class B is not a duplicate for class A and class B can be rename refactored to any other name instead of A as mentioned in section \ref{sec:precon1}. However, in Fig. \ref{figure:fig2} (b), when we try to rename refactor the class B to class Test, we get compile error \textit{"a compilation unit must not import and declare a type with the same name"}~\cite{EclipseWebPage}. 
This is because a class named Test is being imported from another package and on renaming the target class B to Class Test will have compilation error due to duplicate names. 

\begin{figure}[th]
\centering
\begin{minipage}[t]{0.75\linewidth}
\begin{lstlisting}[language=java, basicstyle=\scriptsize\ttfamily,frame=single]
package aftervisitor;
public class Visitor {
  public void print() {
      System.out.println("Hello");
  }
}

package beforevisitor;
import aftervisitor.Visitor;
class A {
    public static void main(String[] args) {
        Visitor v = new Visitor();
        v.print();
    }
}

\end{lstlisting}
\tiny{(a) Before Rename Refactoring Class A}
\end{minipage}
\hfill

\begin{minipage}[t]{0.75\linewidth}
\begin{lstlisting}[language=java, basicstyle=\scriptsize\ttfamily,frame=single]
package aftervisitor;
public class Visitor {
  public void print() {
      System.out.println("Hello");
  }
}

package beforevisitor;
import aftervisitor.Visitor;
class Visitor {
    public static void main(String[] args) {
        Visitor v = new Visitor();
        v.print();
    }
}
\end{lstlisting}
\tiny{(b) After Rename Refactoring Class A to Visitor}
\end{minipage}
\caption{Class already defined in Compilation Unit}
\label{figure:error}
\end{figure}


\begin{figure}[H]
\centerline{\includegraphics[width=85mm,scale=0.5]{CUE.jpg}}
\caption{The Compilation Unit Error}
\label{figure:pic}
\end{figure}

Another example where rename refactoring requires a pre-check if the target name is not already defined in the compilation unit is shown below . In Fig. \ref{figure:error} (a), on executing the output generated is \textit{Hello}. However, after rename refactoring class A to Visitor as shown in Fig. \ref{figure:error} (b), we get compile error as shown in Fig. \ref{figure:pic}.
\\Therefore, it is essential to pre-check that the target class should not have duplicate name with any of imported class after rename refactoring class. 




\label{sec:precon2}

\subsection{If a parent class imports a class from different package, the target child class within same java file cannot be duplicate with that imported class after rename.}

If a parent class imports a class from different package and if we try renaming a child class with the same name as the class imported into its parent class within the same java file, compiler produces an error as \textit{``a compilation unit must not import and declare a type with the same name''}~\cite{EclipseWebPage}.This precondition can be explained by the following example.

\begin{figure}[th]
\centering
\begin{minipage}[t]{0.45\linewidth}
\begin{lstlisting}[language=java, basicstyle=\scriptsize\ttfamily,frame=single]	
A.java

package q;
import p.C;

public class A{	
}

class B extends A{	
}

class D extends B{
}
\end{lstlisting}
\centering{(a) Before}
\end{minipage}
\hfill
\begin{minipage}[t]{0.45\linewidth}
\begin{lstlisting}[language=java, basicstyle=\scriptsize\ttfamily,frame=single]
A.java

package q;
import p.C;

public class A{	
}

class C extends A{	
}

class D extends B{
}	
\end{lstlisting}
\centering{(b) After}
\end{minipage}
\caption{\textbf{RcR from B to C}}
\label{figure:figpc3_1}
\end{figure}

From the above Fig. \ref{figure:figpc3_1}, we see that if a parent class A imports a class C from package `p' and if we try to rename the child class from B to C or D to C, Java compiler produces an error as shown in the Fig. \ref{figure:figpc3_2}. As mentioned in  section \ref{sec:precon2}, the same precondition also holds good for renaming a child class. This precondition is applicable for all ancestor class and we have to trace back and check if any of the parent class is importing a class with same name before renaming the child class within the same java file.
\begin{figure}[htbp]
\centerline{\includegraphics[width=85mm,scale=0.5]{precond3.png}}
\caption{\textbf{Compile error for RcR child class to imported class}}
\label{figure:figpc3_2}
\end{figure}

If a parent class imports a class from different package and if the child class is defined in a separate java file, then in that case we can refactor and rename the child class to the imported class name.\label{sec:precon3}

\section{\textbf{Preconditions of Rename Method Refactoring}}

Rename Method Refactoring (RmR) changes the name of the method and all references to that method to the new name without changing its functionality in the program.

There are three types of Rename Method Refactorings:
\begin{enumerate}
\item Rename Static Method Declarations.
\item Rename Non-static Method Declarations.
\item Rename Constructor Method Declarations.
\end{enumerate}

There are certain preconditions required for RmR which are applicable to each type of RmR.
\begin{enumerate}
	\item The target method cannot be a duplicate of an existing method after rename.
	\item A duplicate method in a child class cannot have different return type.
	\item A duplicate method in a child class cannot reduce visibility.
\end{enumerate}

\subsection {The target method cannot be duplicate of an existing method after rename. }

If two or more methods have same method signature with
below conditions, those methods will be called as duplicate
methods.
\begin{itemize}
	\item Same method name
	\item Same number of parameters
	\item Same parameter types in order
\end{itemize}

From Fig. \ref{fig:RmR}, when we apply RmR from m to n, the compiler produces an error ``\textsl{method n() is already defined in class A}''. This is because another duplicate method already exists in the class. Hence, we have to check the target method name can not be the same as the existing method.

\begin{figure}[th]
\centering
\begin{minipage}[t]{0.45\linewidth}
\begin{lstlisting}[language=java, basicstyle=\scriptsize\ttfamily,frame=single]
class A {

  void m(int a) {	
  }

  void n(int b) {	
  }	
}
 
\end{lstlisting}
\centering{(a) Before}
\end{minipage}
\hfill
\begin{minipage}[t]{0.45\linewidth}
\begin{lstlisting}[language=java, basicstyle=\scriptsize\ttfamily,frame=single]
class A {

  void n(int a) {	
  }

  void n(int b) {	
  }	
}

\end{lstlisting}
\centering{(b) After}
\end{minipage}
\caption{\textbf{Rename Refactoring Method m to n}}
\label{fig:RmR}
\end{figure}

To further define the duplicate methods, some of the use-cases are as follows:

\subsubsection {Duplicate Methods with convertible Input parameters}

From Fig. \ref{fig:RmR4}, method m1 has input parameter type as `int' and m2 has input parameter type as `double'. When we call method m2 with int data type as the parameter, int automatically and implicitly type-casted by java compiler into double and outputs `bye' as shown in Fig. \ref{fig:RmR4}(a). 

When we apply RmR from m2 to m1 and call method m1 by passing `int' as input parameter then the output is `hello' as shown in Fig. \ref{fig:RmR4}(b). Here, it is clearly evident that after RmR, the original output has changed even with pre-checking condition for duplicate method. 	
This implies that the definition for duplicate method is not enough to rely on and we need to revise duplicate methods to prevent these ambiguity.

Below are primitive type coercions (i.e., implicit type conversions).  Java allows to convert primitive types without losing information about a numeric value. 
\textbf{
\begin{center}
byte < short < int < float < double
\end{center}
}

When the duplicate methods have the larger type size and smaller type is passed then it can be converted to larger type size. The revised definition is that the duplicate methods can have the different data type but the property of widening or automatic type conversion has to be taken care of.

\begin{figure}[th]
\centering
\begin{minipage}[t]{0.47\linewidth}
\begin{lstlisting}[language=java, basicstyle=\scriptsize\ttfamily,frame=single]
class A {

  void m1(int i) {	
    System.out.print("hi");
  }

  void m2(double j) {	
     System.out.print("bye");
  }	
  
  void main(String[] args){
	
     // output: bye
     this.m2(100); 
  }
}
 
\end{lstlisting}
\centering{(a) Before}
\end{minipage}
\hfill
\begin{minipage}[t]{0.47\linewidth}
\begin{lstlisting}[language=java, basicstyle=\scriptsize\ttfamily,frame=single]
class A {

  void m1(int i) {	
    System.out.print("hi");
  }

  void m1(double j) {	
     System.out.print("bye");
  }	
  
  void main(String[] args){
	
     // output: hi
     this.m1(100); 
  }
}

\end{lstlisting}
\centering{(b) After}
\end{minipage}
\caption{\textbf{Rename Refactoring Method m2 to m1}}
\label{fig:RmR4}
\end{figure}

\subsubsection {Non-Duplicate methods Ambiguity} 
From Fig. \ref{fig:RmR5}(a), methods m1 and m2 are non-duplicate methods and the output is `hi'. However, when we apply RmR from m2 to m1 as shown in Fig. \ref{fig:RmR5}(b), the compiler produces an error \textsl{``reference to m1 is ambiguous as both method m1(String,float) in A and method m1(Object,int) in A match''}. 

\begin{figure}[th]
\centering
\begin{minipage}[t]{0.48\linewidth}
\begin{lstlisting}[language=java, basicstyle=\scriptsize\ttfamily,frame=single]
class A{

  void m1(String s, float d){
	
      // output: hello
      System.out.print(s); 
   }

  void m2(Object o, int i){
      System.out.print(o);
   }

  void main(String[] args){
      this.m1("hi", 1);
   }
}
\end{lstlisting}
\centering{(a) Before }
\end{minipage}
\hfill
\begin{minipage}[t]{0.48\linewidth}
\begin{lstlisting}[language=java, basicstyle=\scriptsize\ttfamily,frame=single]
class A{

  void m1(String s, float d){
	
      // Error
      System.out.print(s); 
   }

  void m1(Object o, int i){
      System.out.print(o);
   }

  void main(String[] args){
      this.m1("hi", 1);
   }
}

\end{lstlisting}
\centering{(b) After }
\end{minipage}
\caption{\textbf{Rename Refactoring Method m2 to m1}}
\label{fig:RmR5}
\end{figure}

Hence, it is essential to pre-check that the non-duplicate methods on RmR does not result in argument ambiguity.

\subsection{A duplicate method in a child class cannot have different return type.}

If a duplicate method exists in both parent and child class, Java does not allow these duplicate methods to have different return types.

In Fig. \ref{fig:RmR2} (b), when we apply RmR from B.m2() to B.m1(), compiler produces an error \textsl{`` m1() in B cannot override m1() in A as return type int is not compatible with void''}. This is because the duplicate method m1 have different return types which is not allowed. 

\begin{figure}[th]
\centering
\begin{minipage}[t]{0.47\linewidth}
\begin{lstlisting}[language=java, basicstyle=\scriptsize\ttfamily,frame=single]
class A {

  void m1() {
    System.out.print("hi");
  }
}
class B extends A{
 
  int m2() {
    System.out.print("bye");
  }	
}
 
\end{lstlisting}
\centering{(a) Before}
\end{minipage}
\hfill
\begin{minipage}[t]{0.47\linewidth}
\begin{lstlisting}[language=java, basicstyle=\scriptsize\ttfamily,frame=single]
class A {

  void m1() {
    System.out.print ("hi");
  }
}
class B extends A{
 
  int m1() {
    System.out.print ("bye");
  }	
}

\end{lstlisting}
\centering{(b) After}
\end{minipage}
\caption{\textbf{Rename Refactoring Method m2 to m1}}
\label{fig:RmR2}
\end{figure}

\subsection{A duplicate method in a child class cannot reduce visibility.}

The use of access modifiers helps with visible duplicate methods, however there is a proper hierarchy for access modifiers to be followed for duplicate methods.


\begin{center}
\textbf{public $>$ protected $>$ package-public $>$ private}
\end{center}


As shown in \ref{fig:RmR3}(b), the package-public visibility gets reduced to private for duplicate methods m1() and compiler produces an error \textsl{``m1() in B cannot override m1() in A as attempting to assign weaker access privileges; was package''}. 

\begin{figure}[th]
\centering
\begin{minipage}[t]{0.47\linewidth}
\begin{lstlisting}[language=java, basicstyle=\scriptsize\ttfamily,frame=single]
class A {

  void m1() {
    System.out.print ("hi");
  }
}
class B extends A{

  private void m2() {
    System.out.print ("bye");
  }	
}
 
\end{lstlisting}
\centering{(a) Before }
\end{minipage}
\hfill
\begin{minipage}[t]{0.47\linewidth}
\begin{lstlisting}[language=java, basicstyle=\scriptsize\ttfamily,frame=single]
class A {

  void m1() {
    System.out.print ("hi");
  }
}
class B extends A{
 
  private void m1() {
    System.out.print ("bye");
  }	
}

\end{lstlisting}
\centering{(b) After }
\end{minipage}
\caption{\textbf{Rename Refactoring Method m2 to m1}}
\label{fig:RmR3}
\end{figure}

Therefore, it is essential to pre-check that new name of method does not result in visibility reduction. 

\section{\textbf{Preconditions of Rename Field Refactoring}}
Rename Field Refactoring(RfR) changes the declaration and usages of the field to the new name without changing its behavior.
Fields are the variables of a class i.e. instance variables and static variables.
There are certain preconditions required for RfR.

\begin{enumerate}
	\item The target name of field cannot be duplicate with any existing field within same class after rename.
	\item The target name of field cannot be duplicate with any local variable within same method after rename.
\end{enumerate}

\subsection{The target name of field cannot be duplicate with any existing field within same class after rename.}

\textbf{Example 1:} In order to apply RfR on field, we have to pre-check that the new name of the field is not duplicate with any existing field name within same class. As shown in Fig. \ref{figure:sameType}, when we apply RfR from \emph{`j' to `i'}, Java compiler produces an error \textit{``variable i is already defined in class A''}. This is because of the conflict for duplicate field declaration. 

\begin{figure}[th]
\centering
\begin{minipage}[t]{0.4\linewidth}
\begin{lstlisting}[language=java, basicstyle=\scriptsize\ttfamily,frame=single]
public class A {

   int i = 0;
   int j = 2;
}

\end{lstlisting}
\centering(a) Before
\end{minipage}
\hfill
\begin{minipage}[t]{0.4\linewidth}
\begin{lstlisting}[language=java, basicstyle=\scriptsize\ttfamily,frame=single]
public class A {

   int i = 0;
   int i = 2;
}
\end{lstlisting}
\centering(b) After
\end{minipage}
\caption{\textbf{RfR from j to i}}
\label{figure:sameType}
\end{figure}

\textbf{Example 2:} In order to apply RfR on field, we have to pre-check that the new name of the field is not duplicate with any existing field name of any data type within same class. As shown in Fig. \ref{figure:diffType}, when we apply RfR from \emph{`s' to `i'}, Java compiler produces an error \textit{``variable i is already defined in class A''}. This is because the compiler will not be able to distinguish between different type variables `String i' and `int i'. 

\begin{figure}[th]
\centering
\begin{minipage}[t]{0.48\linewidth}
\begin{lstlisting}[language=java, basicstyle=\scriptsize\ttfamily,frame=single]
public class A {

   int i = 0;
   String s = "hi";
}

\end{lstlisting}
\centering(a) Before
\end{minipage}
\hfill
\begin{minipage}[t]{0.48\linewidth}
\begin{lstlisting}[language=java, basicstyle=\scriptsize\ttfamily,frame=single]
public class A {

   int i = 0;
   String i = "hi";
}
\end{lstlisting}
\centering(b) After
\end{minipage}
\caption{\textbf{RfR from s to i}}
\label{figure:diffType}
\end{figure}

\subsection{The target name of field cannot be duplicate with any local variable within same method after rename.}
In order to apply RfR on field, we have to pre-check that the new name of field is not duplicate with any local variable name, if field is being used in the same block or method as local variable. 

As shown in Fig. \ref{figure:sameBlock}(a), on executing the output is `1'. However,  after we apply RfR from \emph{`x' to `y'} and on executing Fig. \ref{figure:sameBlock}(b), the output is `7'. Although the compiler did not produce any error but the behavior of code has changed since the output on RfR got changed. 

\begin{figure}[th]
\centering
\begin{minipage}[t]{0.45\linewidth}
\begin{lstlisting}[language=java, basicstyle=\scriptsize\ttfamily,frame=single]
class A{
   int x = 0;
   void m(int y){
      x++;
      y++;
      System.out.print(x);
   }
   void main(String[] args){
      A.m(5);
   }
}
\end{lstlisting}
\centering(a) Before
\end{minipage}
\hfill
\begin{minipage}[t]{0.45\linewidth}
\begin{lstlisting}[language=java, basicstyle=\scriptsize\ttfamily,frame=single]
class A{
   int y = 0;
   void m(int y){
      y++;
      y++;
      System.out.print(y);
   }
   void main(String[] args){
      A.m(5);
   }
}\end{lstlisting}
\centering(b) After
\end{minipage}
\caption{\textbf{RfR from x to y}}
\label{figure:sameBlock}
\end{figure}

Therefore, it is essential to pre-check that the new name of field is not duplicate with any existing local variable if both are used in same block. 

 \section{\textbf{Preconditions of Rename Local Variable Refactoring}}
 
 
Rename Local Variable Refactoring (RlR) changes the name of the local variable and all references to that variable to the new name without changing its behavior. There are certain preconditions required for RlR.

 \subsection{The target name of local variable cannot be duplicate with any of the existing local variable in a method.}
 
\begin{figure}[th]
\centering
\begin{minipage}[t]{0.45\linewidth}
\begin{lstlisting}[language=java, basicstyle=\scriptsize\ttfamily,frame=single]
public class A {

   void m1(int num) {
	int x = 1;
	int y = 2
    }
}
\end{lstlisting}
\centering(a) Before
\end{minipage}
\hfill
\begin{minipage}[t]{0.45\linewidth}
\begin{lstlisting}[language=java, basicstyle=\scriptsize\ttfamily,frame=single]
public class A {

   void m1(int num) {
	int num = 1;
	int y = 2;
    }
}
\end{lstlisting}
\centering(b) After
\end{minipage}
\caption{\textbf{RlR from x to num}}
\label{figure:precond5_1}
\end{figure}

For example, from Fig. \ref{figure:precond5_1}, when we apply RlR for local variable `x' to `num' , the java compiler produces the error as ``variable num is already defined in method m1(int)''. Similarly we cannot run RlR on `x'  to `y' or `y'  to `num'. Renaming local variables to existing local variable in a method creates a conflict as duplicate local variable.

Therefore, it is essential to pre-check that the target name of local variable should not have duplicate name with the any of the existing local variable in a method after RlR.

\section{\textbf{Preconditions of Rename Package Refactoring}}

Rename Package Refactoring (RpR) changes the name of the package and all references to that package to the new name without changing its behavior.
There are certian preconditions required for RpR.

\begin{enumerate}
	\item The target package cannot be duplicate with any existing package name within same project folder. 
\end{enumerate}

\subsection{The target package cannot be duplicate with any existing package name within same project folder.}

For RpR, we can not use duplicate package name within the same source folder. However, we can run RpR with same package name from different Java Project folders.

\bibliographystyle{IEEEtran}
\bibliography{ref}

\end{document}
