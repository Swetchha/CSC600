\section{\textbf{Preconditions of Rename Field Refactoring}}
Fields are the variables of a class i.e. instance variables and static variables.

Rename Field Refactoring (RfR) changes the declaration and references to the field variables to the new name without changing its behavior.
There are certain preconditions required for RfR.

\begin{enumerate}
	\item The target name of field variable cannot be duplicate with any existing field within same class after rename.
	\item The target name of field variable cannot be duplicate with any local variable within same method after rename.
	\item The duplicate field variable in a child class cannot reduce visibility. 
\end{enumerate}

\subsection{The target name of field cannot be duplicate with any existing field within same class after rename.}

In order to apply RfR we have to pre-check that the new name of the field is not duplicate with any existing field name of any data type within same class. From Fig. \ref{figure:field}(a), when we apply RfR from $j$ to $i$ as shown in Fig. \ref{figure:field}(b), Java compiler produces an error \textit{``variable i is already defined in class A''}. This is because of the conflict for duplicate names for field variables. Similarly when we apply RfR from $s$ to $i$ as shown in Fig. \ref{figure:field}(c), Java compiler produces an error \textit{``variable i is already defined in class A''}. So, irrespective of the data type, we cannot have duplicate names for field variables within same class.

\begin{figure}[th]
\centering	
\begin{minipage}[t]{0.45\linewidth}
\begin{lstlisting}[language=java, basicstyle=\scriptsize\ttfamily,frame=single]
class A {

   int i = 0;
   int j = 2;
   String s = "hi";
}

\end{lstlisting}
\centering(a) Before
\end{minipage}
\hfill
\begin{minipage}[t]{0.45\linewidth}
\begin{lstlisting}[language=java, basicstyle=\scriptsize\ttfamily,frame=single]
class A {

   int i = 0;
   int i = 2;
   String s = "hi";
}
\end{lstlisting}
\centering(b) After Rename Refactoring Field j to i
\end{minipage}

\centering
\begin{minipage}[t]{0.45\linewidth}
\begin{lstlisting}[language=java, basicstyle=\scriptsize\ttfamily,frame=single]
class A {

   int i = 0;
   int j = 2;
   String i = "hi";
}
\end{lstlisting}
\centering(c) After Rename Refactoring Field s to i
\end{minipage}
\caption{\textbf{Rename Refactoring on field variables}}
\label{figure:field}
\end{figure}


\subsection{The target name of field cannot be duplicate with any local variable within same method after rename.}
In order to apply RfR we have to pre-check that the new name of field is not duplicate with any local variable name, if field is being used in the same block as local variable. 

As shown in Fig. \ref{figure:sameBlock}(a), the output is `0'. However, after we apply RfR from $x$ to $y$ as shown in Fig. \ref{figure:sameBlock}(b), the output is `1'. Although the Java compiler do not produce any error but the behavior of code has changed since the output after RfR has changed. This is because if a field variable and a local variable have the same name, then the local variable will be accessed. This process effectively shadows the field variable and therefore known as \textit{Variable Shadowing}. Therefore, it is essential to pre-check that the new name of field is not duplicate with any existing local variable if both are used in same block.
 

\begin{figure}[th]
\centering
\begin{minipage}[t]{0.47\linewidth}
\begin{lstlisting}[language=java, basicstyle=\scriptsize\ttfamily,frame=single]
class A{

   int x = 0;
	
   void m(int y) {
	
      //Output: 0
      System.out.print(x);
   }
	
   void main(String[] args) {
	
      this.m(1);
   }
}

\end{lstlisting}
\centering(a) Before
\end{minipage}
\hfill
\begin{minipage}[t]{0.47\linewidth}
\begin{lstlisting}[language=java, basicstyle=\scriptsize\ttfamily,frame=single]
class A{

   int y = 0;
	
   void m(int y) {
	
      //Output: 1
      System.out.print(y);
   }
	
   void main(String[] args) {

      this.m(1);
   }
}
\end{lstlisting}
\centering(b) After
\end{minipage}
\caption{\textbf{Rename Refactoring Field x to y}}
\label{figure:sameBlock}
\end{figure}

The same concept is applied to class hierarchies. A field variable declared within a parent class will be shadowed by any variable with the same name in a child class. As shown in Fig. \ref{figure:classEx}(a), the output is `0'. However, after we apply RfR on field variable of a child class from $j$ to $i$ as shown in Fig. \ref{figure:classEx}(b), the output is `1'. This Variable Shadowing has changed the original behavior of the program after RfR. 

\begin{figure}[th]
\centering
\begin{minipage}[t]{0.47\linewidth}
\begin{lstlisting}[language=java, basicstyle=\scriptsize\ttfamily,frame=single]
class A {

   int i = 0;
   
   void main(String[] args) {

      B b = new B();
	      
      // Output: 0
      System.out.print(b.i); 
   }
}

class B extends A {

   int j = 1; 
}
\end{lstlisting}
\centering{(a) Before}
\end{minipage}
\hfill
\begin{minipage}[t]{0.47\linewidth}
\begin{lstlisting}[language=java, basicstyle=\scriptsize\ttfamily,frame=single]
class A {

   int i = 0;
   
   void main(String[] args) {

      B b = new B();
	      
      // Output: 1
      System.out.print(b.i); 
   }
}

class B extends A {

   int i = 1; 
}
\end{lstlisting}
\centering{(b) After }
\end{minipage}
\caption{\textbf{Rename Refactoring on child class field variable j to i}}
\label{figure:classEx}
\end{figure}

Therefore, it is essential to pre-check that the duplicate field variables in parent and child class does not result in variable shadowing. 

\subsection{The duplicate field variable in a child class cannot reduce visibility.}

The child class variables are able to reduce the visibility of the parent class variables. The use of access modifiers helps with visible duplicate field variables, however there is a proper hierarchy for access modifiers to be followed for duplicate variables.


\begin{center}
\textbf{public $>$ protected $>$ package-public $>$ private}
\end{center}


\begin{figure}[th]
\centering
\begin{minipage}[t]{0.47\linewidth}
\begin{lstlisting}[language=java, basicstyle=\scriptsize\ttfamily,frame=single]
class A {

   public int i = 0;
	
   void main(String[] args) {
	
      B b = new B();
      
      // Output: 0
      System.out.print(b.i); 
   }
}

class B extends A {

   private int j = 1;
}
\end{lstlisting}
\centering{(a) Before}
\end{minipage}
\hfill
\begin{minipage}[t]{0.47\linewidth}
\begin{lstlisting}[language=java, basicstyle=\scriptsize\ttfamily,frame=single]
class A {

   public int i = 0;
	
   void main(String[] args) {
	
      B b = new B();
      
      // Error
      System.out.print(b.i); 
   }
}

class B extends A {

   private int i = 1;
}
\end{lstlisting}
\centering{(b) After}
\end{minipage}
\caption{\textbf{Rename Refactoring on child class field j to i}}
\label{figure:jtoi}
\end{figure}


As shown in Fig. \ref{figure:jtoi}(a), the two field variables $i$ and $j$ are completely distinct and have completely separate visibilities in different classes. However, on applying RfR on child class variable from $j$ to $i$ as shown in Fig. \ref{figure:jtoi}(b), the public visibility of field variable $i$ gets reduced to private and Java compiler produces an error \textit{``field B.i is not visible''}. 

Therefore, it is essential to pre-check that the new name of the field variable does not result in visibility reduction. 