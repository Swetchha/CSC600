\subsection{The target class file name cannot be duplicate with any existing java file name within same package after rename.}

If classes are defined in separate java files, we have to pre-check that the new name of the target class declared in its class file name is not duplicate with any existing Java file name within the same package. 

\begin{figure}[th]
\centering
\begin{minipage}[t]{0.45\linewidth}
\begin{lstlisting}[language=java, basicstyle=\scriptsize\ttfamily,frame=single]
//A.java

public class A{
}
	
class B{
}


//C.java
//empty file
 
\end{lstlisting}
\centering(a) Before
\end{minipage}
\hfill
\begin{minipage}[t]{0.45\linewidth}
\begin{lstlisting}[language=java, basicstyle=\scriptsize\ttfamily,frame=single]
//C.java

public class C{
}
	
class B{
}


//C.java
//empty file

\end{lstlisting}
\centering(b) After
\end{minipage}
\caption{\textbf{RcR from A to C if A is declared in file A.java}}
\label{figure:classFileName}
\end{figure}

For example, from Fig. \ref{figure:classFileName}, when we apply RcR from \emph{A to C} where C.java is an empty file, the Java compiler produces an error \textit{``Compilation unit `C.java' already exists''}. This is because after RcR the Java file also gets renamed from \emph{A.java to C.java} and therefore creates a conflict for duplicate file name as \emph{C.java} already exists in the same package. 

However, we can apply RcR from \emph{B to C} as class B is not declared within its class file name. 



