\subsection{The target class name cannot be duplicate with any existing java file name within same package after rename.}

If there are more than one Java file within a package, we have to pre-check that the new name of the target class is not duplicate with any existing Java file name within the same package. 

\begin{figure}[th]
\centering
\begin{minipage}[t]{0.45\linewidth}
\begin{lstlisting}[language=java, basicstyle=\scriptsize\ttfamily,frame=single]
//A.java

public class A{}
	
class B{}


//C.java
//empty file
 
\end{lstlisting}
\centering(a) Before 
\end{minipage}
\hfill
\begin{minipage}[t]{0.45\linewidth}
\begin{lstlisting}[language=java, basicstyle=\scriptsize\ttfamily,frame=single]
//C.java

public class C{}
	
class B{}


//C.java
//empty file

\end{lstlisting}
\centering(b) After Rename Refactoring Class A to C
\end{minipage}

\centering
\begin{minipage}[t]{0.45\linewidth}
\begin{lstlisting}[language=java, basicstyle=\scriptsize\ttfamily,frame=single]
//A.java

public class A{}
	
class C{}


//C.java
//empty file

\end{lstlisting}
\centering(c) After Rename Refactoring Class B to C
\end{minipage}
\caption{\textbf{Rename Refactoring Class to existing Java file name}}
\label{figure:classFileName}
\end{figure}

For example, from Fig. \ref{figure:classFileName}(a), when we apply RcR from \emph{A to C} where C.java is an empty file as shown in Fig. \ref{figure:classFileName}(b), the Java compiler produces an error \textit{``Compilation unit `C.java' already exists''}. This is because after RcR the Java file also gets renamed from \emph{A.java to C.java} and therefore creates a conflict for duplicate file name as \emph{C.java} already exists in the same package. 

Also the same concept is applicable to other classes in the same package. When we apply RcR from \emph{B to C} as shown in Fig. \ref{figure:classFileName}(c), the Java compiler produces same error \textit{``Compilation unit `C.java' already exists''}.  



