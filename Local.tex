\section{\textbf{Preconditions of Rename Local Variable Refactoring}}
A local variable is a variable declared inside a method and it is only accessible inside the method that declared it

Rename Local Variable Refactoring (RvR) changes the name of the local variable and all references to that variable to the new name without changing its behavior. There are certain preconditions required for RvR.
\begin{enumerate}
\item The renamed local variable cannot be duplicate with any of the existing local variable in a method or block or constructor.
\item The renamed local variable must not result in variable shadowing.
\end{enumerate}

\subsection{The renamed local variable cannot be duplicate with any of the existing local variable in a method.}
 
From Fig. \ref{figure:precond5_1} and Fig. \ref{figure:precond5_2}, we see two examples of duplicate local variable. In Example 1 when we apply RvR for local variable `x' to `num' , the java compiler produces the error as ``variable num is already defined in method m1(int)''. Similarly we cannot run RvR on  `y'  to `num' or `x'  to `y' . Renaming local variables to existing local variable in a method creates a conflict of duplicity in local variable.

Therefore, it is essential to pre-check that the target name of local variable should not have duplicate name with any of the existing local variable in a method after RvR.

\begin{figure}[th]
\centering
\begin{minipage}[t]{0.45\linewidth}
\begin{lstlisting}[language=java, basicstyle=\scriptsize\ttfamily,frame=single]
public class A {

    void m1(int num) {
       int x = 1; 
       int y = 2
    }
}
\end{lstlisting}
\centering(a) Before
\end{minipage}
\hfill
\begin{minipage}[t]{0.45\linewidth}
\begin{lstlisting}[language=java, basicstyle=\scriptsize\ttfamily,frame=single]
public class A {

    void m1(int num) {
        int num = 1; 
	int y = 2;
    }
}
\end{lstlisting}
\centering(b) After 
\end{minipage}
\caption{\textbf{Rename Refactoring Variable x to num}}
\label{figure:precond5_1}
\end{figure}

\begin{figure}[th]
\centering
\begin{minipage}[t]{0.45\linewidth}
\begin{lstlisting}[language=java, basicstyle=\scriptsize\ttfamily,frame=single]
public class A {

    void m1(int num) {
       int x = 1; 
       int y = 2
    }
}
\end{lstlisting}
\centering(a) Before 
\end{minipage}
\hfill
\begin{minipage}[t]{0.45\linewidth}
\begin{lstlisting}[language=java, basicstyle=\scriptsize\ttfamily,frame=single]
public class A {

    void m1(int num) {
        int y = 1; 
	int y = 2;
    }
}
\end{lstlisting}
\centering(b) After 
\end{minipage}
\caption{\textbf{Rename Refactoring Variable x to y}}
\label{figure:precond5_2}
\end{figure}

\subsection{The renamed local variable must not result in variable shadowing.}
Shadowing refers to the concept of using two variables with the same name within scopes that overlap. When we do that, the variable with the higher-level scope is hidden because the variable with lower-level scope overrides it. This results in the higher-level variable being ``shadowed''. 

\begin{figure}[th]
\centering
\begin{minipage}[t]{0.8\linewidth}
\begin{lstlisting}[language=java, basicstyle=\scriptsize\ttfamily,frame=single]
public class B {

    public int i = 2;
    
    public void display() {
      int j = 1;
      System.out.println(j); 		
   }

    public static void main(String args[]) {
    
      B b = new B();
      b.display();  //Outputs 1
   }
}
\end{lstlisting}
\centering(a) Before 
\end{minipage}
\hfill
\begin{minipage}[t]{0.8\linewidth}
\begin{lstlisting}[language=java, basicstyle=\scriptsize\ttfamily,frame=single]
public class B {

    public int i = 2;
    
    public void display() {
      int i = 1;
      System.out.println(i); 		
   }

    public static void main(String args[]) {
    
      B b = new B();
      b.display();  //Outputs 1
   }
}
\end{lstlisting}
\centering(b) After 
\end{minipage}
\caption{\textbf{Rename Refactoring Variable j to i }}
\label{figure:precond5_3}
\end{figure}


Suppose a local variable has the same name as one of the field variable(instance variable), the local variable shadows the field variable inside the method block. From Fig. \ref{figure:precond5_3}, when we access the variable in the method, the local variable value will be printed shadowing the field variable.

Similarly variable shadowing occurs when the same variables are defined in parent and child classes.
For Example, from Fig. \ref{figure:precond5_4}, after RvR from j to i and k to i ,when we try to access the variable `i' in methods through Class B and C, the respective local variable will be printed shadowing the field variable of its parent class.


\begin{figure}[th]
\centering
\begin{minipage}[t]{0.8\linewidth}
\begin{lstlisting}[language=java, basicstyle=\scriptsize\ttfamily,frame=single]
public class B {

    int i = 1;

    public void m1() {
      int j = 2;
      System.out.println(j);
   }
}

class C extends B {

    int k = 3;
}

public class Main {

    public static void main(String args[]) {
  
       B b = new B();
       b.m1();     // outputs 2
       C c = new C();
       c.m1();     // outputs 2

   }
}
\end{lstlisting}
\centering(b) Before 
\end{minipage}
\hfill
\begin{minipage}[t]{0.8\linewidth}
\begin{lstlisting}[language=java, basicstyle=\scriptsize\ttfamily,frame=single]
public class B {

    int i = 1;

    public void m1() {
      int i = 2;
      System.out.println(i);
   }
}

class C extends B {

    int i = 3;
}

public class Main {

    public static void main(String args[]) {
  
       B b = new B();
       b.m1();   // outputs 2
       C c = new C();
       c.m1();   // outputs 2
   }
}
\end{lstlisting}
\centering(b) After 
\end{minipage}
\caption{\textbf{Rename Refactoring Variables j to i  and k to i }}
\label{figure:precond5_4}
\end{figure}


Therefore, it is essential to pre-check that local variable must not result in variable shadowing after RvR.